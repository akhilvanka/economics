\newpage
\chapter{\normalfont Price Elasticity of Demand}
Elasticity is a concept that is going to be used extensively in econ. You will be expected to apply all the concepts in this chapter onto your graphs both in class and on the AP exam. 
\lecture{11}{mon 11 oct 10:30}{Inefficiency}
\section{Property Rights \& Market Failure}
\begin{definition}
    \textbf{Market Failure} is whenever DWL is current. However in this scenario we will also be covering when the equilibrium quantity of output is different than the socially optimal level of a good. 
\end{definition}

I mentioned the term socially optimal, and that just means that the Marginal Social Benefit = Marginal Social Cost. Social is a bit different then the free market that we always use on our graphs. For now, just keep this concept in mind, as it will come in later units near the end of microecon.

Same with the idea of externality, or external influence. There can be both Negative and Positive Externality. 
\begin{itemize}
    \item Negative Externality
        \begin{itemize}
            \item When the production or consumption of goods creates spillover costs
            \item The cost is something that neither party is involved with, some examples of this are pollution, smoking, and drinking
        \end{itemize}
    \item Positive Externality
        \begin{itemize}
            \item The opposite of negative externality, but instead of a cost its a benefit
            \item Examples are education, vaccinations, and mass transit
        \end{itemize}
\end{itemize}

Property rights are also linked to externality, and its just the idea that if you own a resource or good, you have incentive to take care of it. However, when property rights are not given/are unclear, you may want to use that good quickly or not care for it at all. 

Dont worry about this chapter too much, its more of things to keep in mind for later units of econ. 
