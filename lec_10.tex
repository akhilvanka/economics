\newpage
\chapter{\normalfont Consumer Choice and Utility Maxmimization}
Just a heads up, this unit was taught very weirdly by my teacher, so I will attempt to use what they taught and also put some of the concepts into my own words. Once again, sorry for any mistakes that could come up. 
\lecture{10}{fri 8 oct 7:35}{Utility}
\section{Marginal Utility}
\begin{definition}
    Utility is how much satisfaction you get from the consumption of a good or service. Utility is not usefulness, and is subjective. This causes utility to be difficult to quantity. 
\end{definition}
Within microeconomics, economists have to often conduct \textbf{Utility Analsys} to understand how people behave rather than how they think. One of the key theories when conducting utility analysis is the \textbf{Theory of consumer choice} which states that each consumer will spend their income in a way that will yield the greatest satisfaction.
Within the idea of utility itself, there exists two forms: 
\begin{itemize}
    \item Total Utility, or the total benefit to a consumer from all the units of a good purchased
    \item Marginal Utility, or the additional benefit from one additional unit of a good purchased. Another way to think of this is the change in total utility from the purchase of 1 more unit of a good. 
\end{itemize}
Its important to realize that that if you purchase a high number of goods, the total utility will be high but the marginal utility will be low.
\[
    \uparrow \text{number of goods purchased} \to \uparrow \text{Total utility} + \downarrow \text{Marginal utility}
.\] 
\textbf{Total and Marginal Utility}
\begin{table}[!h]
    \begin{center}
        \begin{tabular}{l|l|l}
            \toprule
            \textbf{Tacos consumed per meal} & \textbf{Total Utility} & \textbf{Marginal Utility}\\
            \midrule
            0 & 0 & \empty \\
            1 & 10 & 10 \\
            2 & 18 & 8 \\
            3 & 24 & 6 \\
            4 & 28 & 4 \\
            5 & 30 & 2 \\
            6 & 30 & 0 \\
            \textcolor{red}{7} & \textcolor{red}{28} & \textcolor{red}{-2} \\
            \bottomrule
        \end{tabular}
        \caption{Utility Chart}
        \label{tab:utility}
    \end{center}
\end{table}

So lets focus on the marginal utility section of the table. Notice the diminishing marginal utility that is occuring from 10 to -2. This idea will come in handy later, but for now just see that Marginal Utility is \textbf{the increase in total utility}. In this example, when someone eats two tacos, they gain a marginal utility of 8 for that 2nd taco. $18 - 10 = 8$ 

This trend is called the \textbf{Law of Diminishing Marginal Utility}, where the added satisfaction declines as a consumer acquires additional units of the good. A good example of this is purchasing cars. If you have a strong desire for a car, what happens to that desire if you buy a second car, or a third one. It becomes unneccesary. 

Now how can we use marginal utility and the law previously mentioned, well we can look to the \textbf{Optimal Purchase Rule} which states that people will buy the quantity of each good at which price and marginal utility are equal. So consumers will(or should stop) purchase any good that has a price equal to the marginal utility they gain from it. With this comes a weird statement, but can be summarized to if the marginal utility is greater than or less than the price, then the consumer can improve their well being/purchase optimally by purchasing more or less.
\begin{itemize}
    \item[!] Just note that you want price to equal marginal utility, so try to match them, but if its not possible, use the greater value. What I mean is, lets say a good is \$10 but gets me 11 units of marginal utility, but if I purchase one more, it gets me 9 units of marginal utility, go for the quantity that has a higher marginal utilty. 
\end{itemize}

\textbf{Marginal Utility and Demand}
\begin{itemize}
    \item Utility has a connection to the slope of the Demand Curve
        \begin{itemize}
            \item Law of diminishing marginal utility $\implies$ negative slope of demand curves
            \item $\uparrow$ price $\implies$ $\downarrow$ quantity demanded $\implies$ $\uparrow$ marginal utility
                \begin{itemize}
                    \item[!] Dont worry if this does not make any sense, cause in my opinion it does not. When price goes up, quanity demanded goes down, this we know. But why does that increase marginal utility? (I will ask other teachers and update this when I get answers)
                \end{itemize}
        \end{itemize}
\end{itemize}

\textbf{Theory of Consumer Behavior}\\
Ah this topic, one that also was not taught at all. I do recommend everyone taking this course does do external activities. But regardless, lets assume a consumer is rational (they want to maximize their total utility), they have preferences, budget constraints, and every good has a price. So, every consumer has to make choices. For every choice, they forgo something else. Sounds similar to something we know, Opportunity Costs. In this case we are measuring the OC of the utilty. 

\textbf{Utility Maxmizing Combination}\\
Lets assume there is a \$10 budget, and I want to purchase two goods, dubbed Product A and Product B. Product A which will be in red is \$1, while Product B which is in black is \$2. 

\begin{table}[!h]
    \begin{center}
        \begin{tabular}{l|l|l|l|l}
            \toprule
            Unit of product & \textcolor{red}{Marginal Utility} & \textcolor{red}{MU/\$} & Marginal Utility & MU/\$ \\
            \midrule
            First & 10 & \textcolor{teal}{10} & 24 & \textcolor{violet}{12}\\
            Second & 8 & \textcolor{teal}{8} & 20 & \textcolor{teal}{10} \\
        Third & 7 & 7 & 18 & \textcolor{violet}{9} \\
            Fourth & 6 & 6 & 16 & \textcolor{teal}{8} \\
            Fifth & 5 & 5 & 12 & 6 \\
            Sixth & 4 & 4 & 6 & 3 \\
            Seventh & 3 & 3 & 4 & 2 \\
            \bottomrule
        \end{tabular}
        \caption{Allocation of money}
        \label{tab:incomeallocation}
    \end{center}
\end{table}

Alright, the table looks way more intimidating then the process really is. When comparing OC of marginal utility, it is recommended to compare the utility per price. So divide all the marginal utility by the price of the good and record them, as I did and created a MU/\$ section. Then just pick the larger Marginal Utility per dollar. The teal number is a unique condition, when the MU/\$ is the same between the products, "buy" both of them.

This is called the \textbf{optimal consumption bundle} is the consumption bundle that maximizes a consumer's total utility, given the budget constraints. 
\[
    \frac{\text{\textcolor{red}{MU of product A}}}{\text{\textcolor{red}{Price of A}}} = \frac{\text{MU of product B}}{\text{Price of B}}
.\] 

In this scenario, 
\[
    \frac{\text{\textcolor{red}{8 Utils}}}{\text{\textcolor{red}{\$1}}} = \frac{\text{16 Utils}}{\text{\$2}}
.\]

This is just a mathematical way of representing it, but I believe using a table is the quickest way to solve this. 
