\lecture{1}{fri 10 sep 8:35}{Introduction to Microeconomics}
\newtheorem{definition}{Definition}
\begin{itemize}
    \item Teacher: \ldots 
    \item Books:
        \begin{itemize}
            \item Microeconomics: Krugman's Economics for AP
            \item Macroeconomics: Unknown
        \end{itemize}
    \item Period 2, make sure to check daily block rotations for when class is
    \item AP Exam:
        \begin{itemize}
            \item Microeconomics: Details to be given at a later date
            \item Macroeconomics: Details to be given at a later date
        \end{itemize}
    \item No mandatory homework, but use the workbook in order to gain practice
\end{itemize}
\newpage

\chapter{\normalfont Introduction}
\begin{definition}
    Economics is the study of scarcity and choices. Economics will study the choices of individuals, firms, societies, and governments
\end{definition}
Economics assumes that society has unlimited wants and limited resources, and because of this choices must be made. The cost of every choice becomes the opportunity cost. Since everyone acts in their own "self-interest", everyone makes decisions based on the \textbf{marginal cost} and the \textbf{marginal benefits} of the choice. 
\section{\normalfont 4 Factors of Production}
Since all resources are scare, they can be catagorized into: 

Land: Natural resources that are used to produce goods and services.
\begin{itemize}
    \item Timber
    \item Clean Water
    \item Minerals
\end{itemize}

Capital: Physical capital, where any human resource that is used to create goods and services, as well as Human capital which the skill or knowledge gained by a worker through education and/or experience. 
\begin{itemize}
    \item Machinery 
    \item Building
    \item Tools
    \item Worker training
\end{itemize}

Labor(Human): The effort a person devotes to a task for which they are paid for.
\begin{itemize}
    \item Effort of workers
        \begin{itemize}
            \item[!] Do not confuse this with human capital
        \end{itemize}
\end{itemize}

Entrepreneurship: Creative leaders who combine the other factors of pridyction to create goods and services.
\begin{itemize}
    \item Risk Taking
    \item Innovation
    \item Creative combination of resources to increase production
\end{itemize}

