\lecture{3}{tues 14 sep 1:29}{Advantages}
\section{Absolute and Comparative Advantage}

Countries will vary in their absolute and comparative advantages because they have different factors of production. This leads them to have different \textbf{Absolute and Comparative advantages} then other countries. 
\begin{definition}
    Absolute Advantage is who can make the \textbf{MOST} with the given resources. They have the highest output or have a more efficient use of scarce resources(fewest resources required to produce).
\end{definition}
\begin{definition}
    Comparative Advantage is who has the lowest opportunity cost to produce. Who has to give up less in order to produce something. 
\end{definition}

In order to answer a advantage question, you must been given data about \textbf{TWO} entities(individuals/firms/countries) and \textbf{TWO} outputs or inputs. It can be presented in classic data tables, a sentence, or a PPC curve/graph. A data table is the best way to orient the data for the advantage calculations. 

\textbf{How to determine absolute and comparative advantage}
\begin{enumerate}[label=Step \arabic*:]
    \item If you are not given a data table, its recommended to make a table and leave space so you can write notes down
    \item Pay attention, check if the data is provided is about \textbf{inputs}(resources to create one good) or \textbf{outputs}(number of goods produced)?
    \item Write whatever conclusion you have at Step 2
    \item Read the data table that you have or created and see who has absolute advantage in each alternative
    \item Calculate the opportunity cost for each alternative, and compare the values to determine who has comparative advantage. 
\end{enumerate}
\begin{example}
    Determining Absolute Advantage, when the data is \textbf{Outputs}
    \begin{table}[h!]
        \begin{center}
            \begin{tabular}{l|l|l}
                \toprule
                \textbf{Output} & \textbf{Bikes} & \textbf{Cars}\\
                \midrule
                U.S. & 4 & 2\\
                Japan & 5 & 1\\
                \bottomrule
            \end{tabular}
            \caption{Production Possibilities between Bikes and Cars}
            \label{tab:table2}
        \end{center}
    \end{table}

Imagine the data as two seperate PPC curves, and reason your way through:
\begin{itemize}
    \item Using all given resources, the US can either produce 4 bikes \textbf{OR} 2 cars. 
    \item Using all given resources, Japan can either produce 5 bikes \textbf{OR} 1 car.
\end{itemize}

Absolute advantage looks for who can create the \textbf{MOST} outputs with the given resources:
\begin{itemize}
    \item Japan has the absolute advantage in producing bikes because 5 bikes is greater than 4 bikes. 
    \item The US has the absolute advantage in producing cars because 2 cars is greater than 1 car. 
\end{itemize}
\end{example}
\begin{example}
    Determining Absolute Advantage, when the data is \textbf{Inputs}
    \begin{table}[h!]
        \begin{center}
            \begin{tabular}{l|l|l}
                \toprule
                \textbf{Input} & \textbf{Dishwashers} & \textbf{T.V.s}\\
                \midrule
                Spain & 4 & 2\\
                China & 5 & 1\\
                \bottomrule
            \end{tabular}
            \caption{Production Possibilities between Dishwashers and T.V.s}
            \label{tab:table3}
        \end{center}
    \end{table}

The numbers in this chart are \textbf{inputs} rather than outputs, and represents the \textbf{numbers of hours} to produce either good in both countries. The reasoning changes when you are working with the number of hours that are required to produce one good than the outputs. 

Look for who can create \textbf{ONE} product with the \textbf{FEWEST} resources:
\begin{itemize}
    \item Spain has the absolute advantage in producing dishwashers because 4 hours < 5 hours
    \item China has the absolute advantage in producing TVs because 1 hour < 2 hours
\end{itemize}
\end{example}
\textbf{Comparative Advantage}
Comparative advantage follows the same setup as Absolute advantages, but requires a bit more work in order to solve. The goal for Comparative advantage is to look for \textbf{who gives up the least}.
\begin{example}
    Calculating Comparative Advantage, when the data is \textbf{Outputs}
    \begin{table}[h!]
        \begin{center}
            \begin{tabular}{l|l|l}
               \toprule
               \textbf{Output} & \textbf{Bikes} & \textbf{Cars}\\
               \midrule
               U.S. & 4 & 2\\
               Japan & 5 & 1\\
               \bottomrule
            \end{tabular}
            \caption{Production Possibilities between Bikes and Cars}
            \label{tab:table4}
        \end{center}
    \end{table}

The numbers in the chart are outputs, and represent the maximum number of goods that can be produced in each country with a given amount of resources. 
\begin{itemize}
    \item Every time the U.S. produces \textbf{ONE} bike, they give up $\frac{1}{2}$ cars.  
        \begin{itemize}
            \item[!] Remember, you are trying to find out what they give up for one unit, so divide both sides by 4 bikes to get how many cars they give up.
        \end{itemize}
\end{itemize}

Once you find out the OC for producing each good, its as simple as choosing the lower OC:
\begin{itemize}
    \item The U.S. has a \textbf{comparative advantage} in Cars, and should specialize in their production. The U.S. can produce a auto for the OC of 2 bikes which is less than Japans OC of 5 bikes. 
    \item Japan has a \textbf{comparative advantage} in Bikes, and should specialize in their productionl. Japan can produce a bike for the OC of $\frac{1}{5}$ an auto, which is less than the U.S.s OC of $\frac{1}{2}$
\end{itemize}
\end{example}
An important point to think about is that if each country specializes in their comparative advantage, and engages in trade, they can both \textbf{CONSUME} outside their PPC.
\begin{example}
    Calculating Comparative Advantage, when the data is \textbf{Inputs}
    \begin{table}[h!]
        \begin{center}
            \begin{tabular}{l|l|l}
                \toprule
                \textbf{Input} & \textbf{Dishwashers} & \textbf{T.V.s}\\
                \midrule
                Spain & 4 & 2\\
                China & 5 & 1\\
                \bottomrule
            \end{tabular}
        \end{center}
        \caption{Production Possibilities between Dishwashers and TVs}
        \label{tab:table5}
    \end{table}

Same deal with Absolute Advantage, but we need to think differently to calculate OC because its an input problem.
\begin{itemize}
    \item When Spain produces \textbf{ONE} dishwasher, it must use 4 hours to do so, which means Spain gives up producing 2 TVs
        \begin{itemize}
            \item[!] Remember, the data is already in terms of \textbf{ONE} unit, this means to calculate Comparative Advantage, divide the dishwashers by the number of hours it takes to produce TVs. The OC of producing 1 dishwasher is $4/2$ or 2 TVs.
        \end{itemize}
\end{itemize}

Once you calculate the Advantages, you can compare the OC and choose whichever one is a smaller value:
\begin{itemize}
    \item Spain has \textbf{comparative advantage} producing dishwashers. 
    \item China has \textbf{comparative advantage} producing TVs.
\end{itemize}
\end{example}

\textbf{Quick Notes:}
International trade leads to mutual gain as it allows each country to specialize more fully in the producion in what it does best according to the \textbf{law of comparative advantage}.
