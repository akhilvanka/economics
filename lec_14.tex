\newpage
\chapter{\normalfont Other Elasticities and Taxes}
This chapter includes the other forms of elasticities and how to calculate them.
\lecture{14}{fri 14 oct 12:00}{Elasticities}
\section{Other Elasticities}
So this chapter covers some of the other forms of elasticity. The first one we will cover is income elasticity. Income elasticity is the, 
\[
    \frac{\%\Delta QD}{\%\Delta Income}
.\] 

This covers changes in income, however the change in income will cause a change in demand as a whole. This results in a curve shift, and measures how responsive demand is to a change in income. Income elasticity depends on if the good is a normal good or inferior good. With inferior goods, If $E_Y < 0$, then Income has risen and Quantity Demanded drops, or the inverse. With normal goods, $E_Y$ cannot go lower than 0. If $E_Y > 0$, then Income has risen and Quantity demanded has increased, or the inverse. If it is a necessity, then $0 < E_Y < 1$ and if it is a luxury good $E_Y > 1$.

That is what income elasticity is, and now we can move onto Cross-Price Elasticity. Cross price is used for goods that are related in some way and is, 
\[
    \frac{\%\Delta QD of Good X}{\%\Delta P of Good Y}
.\] 

A change in price Y can cause a change in demand X. This can also create a curve shift with demand, and measures how far does the demand curve shift. If the goods are complements, then price of Y will increase, QD of X will decrease and $E_{CP} < 0$. If they are substitutes, then price of Y increase, and QD of X increases, and  $E_{CP} > 0$. If they are unrelated, then price of Y increases, and QD of X does not change, and $E_{CP} = 0$. The inverse for substitutes and complements are also valid.

We can also calculate Price Elasticity of Supply. Instead of measuring consumer response to a change in price, it measures a producers' response to a change in price. It can also be calculated the same as demand, so I will not go to far into depth with it. Just make sure to do some practice problems involving all of these. 
